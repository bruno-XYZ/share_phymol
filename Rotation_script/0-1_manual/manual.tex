\documentclass[parskip]{scrartcl}
%\usepackage{xcolor}
%\definecolor{bgcol}{rgb}{0.7,0.7,0.75}
\usepackage{minted}
\usemintedstyle{native}
%\setminted{bgcolor=bgcol}

%% Write dihed file from ang_def file

\newcommand\help{(}

\title{\Large Documentation for the coordinate transformation script}
\date{\normalsize Version from \today}
\author{\normalsize Author: Bruno von Brüning (\texttt{bruno.vonbruning@student.kuleuven.be})}

\begin{document}\maketitle

\section{Informations}

    \subsection{Problem}
        If a property of a molecule depends on the conformation of this molecule, multiple
        minima have to be considered.
        To find these minima multiple starting geometries have to be created to find 
        all the possible minima. Depending on the molecule and the geometries to be considered
        , i.e. different dihedrals angles a multitude of structures may have to be genereated.
        Such displacement like change of dihedral will be called \textit{operations} in the
        following. 
        This can be a tedious and error-prone labor. The author provides a script that
        mostly automizes this process to enable you an easier and more efficient work flow.
        At the moment the script is limited to rotation around dihedrals but is designed to
        feature more operations. Please contact the author if you want to propose extension of
        the current script.
        Except the manual notation of the operations and values to be considered the user of
        this script will face the minimal work of executing this script and submitting
        calculations to his or her cluster.

    \subsection{Structure of the script}
        The main purpose of the provided code is to rotate a molecule around dihedral angles 
        in order to obtain input structures for subsequent quantum-mechanical calculations.
        More transformations like changing the conformation could be implemented in the future
        for interest in this please contact the author
        (\texttt{bruno.vonbruening@student.kuleuven.be}).
        Next to this main feature a upstream script to permute through possible combinations 
        of angle values is provided, the resulting combination of multiple operations is
        called transformation job. The script generates job file where the operations for the
        multiple displacement to be executed are saved (\texttt{transformation file}). 
        Find more information about this script in section \ref{op-sec}.

        The subsequent main script performs the transformations provided in the 
        \texttt{transformation file} on coordinates of a
        molecule coordinated provided in example in a \texttt{.xyz} file. The script outputs
        all the resulting structures as coordinate files or input files for computational
        chemsitry software. At the moment only \texttt{Gaussian} input format is supported.
        Please contact the author if other input formats may be implemented.
        Find more information about this scipt in section \ref{main-sec}

        The resulting calculations can be automatically evaluated by the 
        \texttt{evaluate\_output\_script}. Resulting angles, energy and dipolemoment are read
        out and either written in a \texttt{.csv} or a \texttt{.txt} format. Identical
        resulting can conformations are indentified by the same resulting angles. 
        Currently only \texttt{Gausssian} \texttt{.log}
        files can be read. Please contact the author if you desire other formats to be read.

    \subsection{Technical information and tipps}
        The three provided script are saved in individual folders. Modules are shared between
        these scripts and saved in the folder \texttt{modules}. The scripts access these
        modules by a relative path, hence please do not change the level of the
        \texttt{module} folder or also change the relative path in the scripts.
        A help entry for all of the following python scripts can be called
        by calling the respective
        funtion with the option \texttt{-h}.
        \begin{minted}{bash}
        ./script_[...] -h
        \end{minted}



\section{Generate the operation file}
    \subsection{Usage}
        The script \texttt{script\_ang-to-op.py} takes a \texttt{input} file with 
        the definition of    the possible operations and
        values and writes all the possible combintations of these values in an 
        \texttt{output} file. By default the name of the output file while be name of the
        inputfile truncated at te extension or the occurence of \texttt{\_ang} and get the
        ending \texttt{\_op.txt} appended.

        \begin{minted}{bash}
        ./script_ang-to-op.py [INPUTFILE]
        \end{minted}

        Following further options can be chosen

        \begin{tabular}{lll}
        Options & Purpose & example
            \\\texttt{-ofilename}     & Name for the outputfile to be generated
            & \texttt{my\_output\_file.txt}
        \end{tabular}
    \subsection{Inputfile format}
        The format of the \texttt{input}-file goes in the following order:

        \def\lenone{0.5\textwidth}%
        \def\lentwo{0.3\textwidth}%
        \begin{tabular}{lll}
            Entry & Description & Example
            \\\hline Comments
                &\parbox[t]{\lenone}{
                comments are indicated with "!" or "!!"
                \vspace{.5\baselineskip}
                }
                & \texttt{[command] ! [comment]}
            \\\hline First line
                &\parbox[t]{\lenone}{
                a range of atoms for example \texttt{\{1,...,5,10,..13,17,..20\}} where 
                \texttt{-} indicates ranges and \texttt{,} a list
                \vspace{.5\baselineskip}
                }
                & \texttt{1-5,10-13,17-20}
            \\\hline Second line
                &\parbox[t]{\lenone}{
                First a operation type (\texttt{ROT} or \texttt{FIX} for fixing atoms for the optimization (givin
                the \texttt{-1} as parameter)). Then separated by a
                space the following option are given.\\
                \texttt{ROT} takes four numbers for the labels of the atoms in the dihedral of interest\\
                \texttt{FIX} takes a list of atoms in the same format as above.
                \vspace{.5\baselineskip}
                }
                &
                \parbox[t]{\lentwo}{\texttt{ROT 2,3,4,5} \\ \texttt{FIX 30-40,50-60}
                }
            \\\hline(Third line)
                &\parbox[t]{\lenone}{
                \texttt{FIX} option: blank or omitted\\
                \texttt{ROT} option: values of desired angles of dihedral as a list, prefix of \texttt{pm} if plus
                and minus these values are desired
                \vspace{.5\baselineskip}
                }
                &\parbox[t]{\lentwo}{
                For \texttt{ROT}:  \texttt{pm 15,75} or \texttt{180}
                }
            \\\hline Separation line
                &\parbox[t]{\lenone}{
                Between different entries there has to be an empty line (empty before possibly
                comments)}
                &\parbox[t]{\lentwo}{
                \texttt{"  \textbackslash n"}, \texttt{" ![!] [...]\textbackslash n"}
                }
        \end{tabular}

        An arbitrary number of these objects these objects can be included in the program in
        arbitrary order. An example can be found in \ref{def_file}
        
        \begin{listing}
        \caption{Format of inputfile for operation file generation\label{def_file}}
            \begin{minted}{text}
            !! angle alpha
            204-219
            ROT 3,2,204,205
            pm 15

            !! angle beta
            214-219
            EXC

            [...]
            \end{minted}
        \end{listing}
    \subsection{Format of operations file \label{op}}

    The ordering of the operations in the operations file will impact the filename that will
    be produced.

    Operations are  

    The entries are separated by blank lines. The first and last line do not have 
    to be a blank line.
    Notably lines with only comments also account as blank lines.

\section{Generate the \texttt{Gaussian} input files}
    \subsection{Purpose}
        The script reads the operation from the the operations file (see \ref{op}) and performs
        these operations on the coordinates provided within a molecule coordinate file. 
        Either input files or molecule
        coordinate files can chosen to be generated (first need a further file with the head of
        the calculation as furthe input, the body file). These files will be named according t
        the
        corresponding values given in the operations file and saved within an own folder. The
        folder will get 
    \subsection{Usage}
        The Displacment of the coordinates requires a file with input coordinates 
        (at the moment only \texttt{.pdb} format) 
        and the file containing the operations (See subsection \ref{op}).
        Furthermore, a body file is required
        if \texttt{.com} files are to be generated (find exmpale in listing \ref{hfile}.
        %Further options can be found in table \ref{Gen_in}.
        The basic execution and further options can be found below.
        \begin{minted}{bash}
            ./script_displacement [molecule.pdb] [operations.txt]
        \end{minted}
            \begin{tabular}{lll}
                Options & Purpose & example
                \\\texttt{-oformat}     & outputformat of transfomred coordinates
                & \texttt{XYZ}, \texttt{PDB}, \texttt{COM}(Default)
                \\\texttt{-oname}         & stem for folder and filenames to be used
                & \texttt{}
                \\\texttt{-bfile}       & headerfile, required to generate \texttt{.com} files
                & \texttt{body\_file.txt}
            \end{tabular}

    \subsection{\texttt{-oformat}: Output format}
    The format of the coordinates to be printed can be chosen with the option \texttt{-oformat}.
    Current accepted values are \texttt{XYZ}, \texttt{PDB}, \texttt{COM} that correspond
    respectively to \texttt{.xyz}, \texttt{.pdb}, and \texttt{.com} format.
    If the \texttt{.com} format is chosen a body file has to be provided by envoking the option
    \texttt{-bfile}. 

    \subsection{\texttt{-bfile}: body file}
    To obtain input files for \texttt{Gaussian} a body with informations for this calculation has to
    be provided to which the calculated coordinates are going to be appended.
    An example of such a headerfile is provided in listing \ref{hfile}.
    Lines starting with \texttt{!} will be recognized to be replaced with information corresponding
    to the current file. The flag \texttt{!chkname} will be replaced with the specific filename.
    The flag \texttt{!jobname} will be replaced with with the filename without the file extension.
    Following is an example for such a header file.
    \begin{listing}
    \caption{Example of a body file\label{hfile}}
    \inputminted{text}{listing_sources/header_file.txt}
    \end{listing}


    \subsection{\texttt{-oname}: Outputfile name}
    The stem -- the first part of the names of the following files -- 
    can be determined with the option
    \texttt{-oname}. The default is the filename of the input file cut after the first occurence of
    a \texttt{.} character. In example, an input filename like \texttt{[path]/molecule.pdb}
    would give the default stem \texttt{molecule}.
    A folder for the output files is created with the name of the \texttt{stem} and appended
    \texttt{\_\{oformat\}}, where the outputfiles are saved. 
    The outputfiles themselves will be named
    by the \textit{stem} and the appendices \texttt{\_\{values\}} for all the values given for this
    operation in ordering of the outputfile. The file extension is the same as \texttt{omode}. For
    example, with the \textit{stem} molecule and provided values 90, 180 and the chosen outut format
    \texttt{.xyz} will lead to the filename \texttt{molecule\_90\_180.xyz}. 
    Notably, these values are
    casted to integers if float are given inside the operations file.

\section{Utilities}
    \subsection{\texttt{chk\_copy.sh} script}
        This script transfers checkpoint files from a provided folder to the present working
        directory where the \texttt{.com} files building up on these \texttt{.chk} files may
        be saved. 
        The name of the \texttt{.com} files is truncated after
        given regular expression and matched with the \texttt{.chk} files. 
        In example, giving the script \texttt{B3} as second option will let the
        script search for the \texttt{Mol\_0\_90\_15\_B3LYP.com} 
        for a file matching \texttt{Mol\_0\_90\_15*.chk*} in the folder provided in the first
        argument. This file is then copied to the present working directory and given the name
        \texttt{Mol\_0\_90\_15\_B3LYP.chk}
        \begin{minted}{bash}
            ./chk_copy.sh [PATH_to_.chk_files] [string_to_cut_before]
        \end{minted}
\clearpage\section{Evaluate the \texttt{Gaussian} output files}
    
\end{document}
